\documentclass{standalone}
\usepackage{tikz}
\usetikzlibrary{calc}

\begin{document}
\begin{tikzpicture}
    % Define points
    \coordinate (A) at (0,0);
    \coordinate (B) at ({16*cos(45)},{16*sin(45)});
    \coordinate (C) at ({16*cos(45) + 6*cos(30)}, {16*sin(45) + 6*sin(30)});

    % Draw points
    \filldraw (A) circle (2pt) node[below left] {A(0,0)};
    \filldraw (B) circle (2pt) node[above left] {B};
    \filldraw (C) circle (2pt) node[above right] {C};

    % Draw lines between points
    \draw[thick] (A) -- (B) -- (C);

    % Draw force L at point C, vertically downward, black and thicker than members
    \draw[->, ultra thick, black] (C) -- ++(0,-3) node[below] {$L$};

    % Draw force G_1 at the midpoint of AB, vertically downward
    \coordinate (M) at ($ (A)!0.5!(B) $);
    \draw[->, ultra thick] (M) -- ++(0,-3) node[below] {$G_1$};

    % Draw force G_2 at the midpoint of BC, vertically downward
    \coordinate (N) at ($ (B)!0.5!(C) $);
    \draw[->, ultra thick] (N) -- ++(0,-3) node[below] {$G_2$};

    % Draw force T at point B, direction -145 degrees from x-axis
    \draw[->, ultra thick] (B) -- ++({3*cos(-145)},{3*sin(-145)}) node[left] {$T$};

    % Draw force A_x at point A, along positive x-axis
    \draw[->, ultra thick] (A) -- ++(3,0) node[right] {$A_x$};

    % Draw force A_y at point A, along positive y-axis
    \draw[->, ultra thick] (A) -- ++(0,3) node[above] {$A_y$};

\end{tikzpicture}
\end{document}